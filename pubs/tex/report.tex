\title{Community Health Worker Project}
\maketitle

In December 2008, the Vodafone's Social Investment Fund agreed to support 
the Community Health Worker Project in South Africa. Funding was awarded to 
GeoMed (Pty) Ltd to develop and implement a system for 
monitoring and evaluating the services and associated outcomes of community 
health workers in South Africa. The technology relied heavily on the use of 
a cellphone-based application by the community health workers. The cellphone 
application would allow the community health worker to record the service 
rendered to each client and the health status of the client. This report 
is summarises the development of the project to date. 

\subsection*{Community-based Health Care in South Africa}
 
Community-based health care has been operating to to some degree for the 
past 15 years. However, due to the dramatic rise in disease, such as 
Tuberculosis and HIV/AIDS and the increased demand for medical resources, 
the use of Community Care Givers (CCG) as an intervention within primary
health care delivery programme has increased.  In more recent years, 
the South African government has tried to formalise community-based 
services. The Government's rationale is that an adequately supported CCG 
programme is a cost-effective means to firstly provide the necessary 
care for certain chronic conditions and, secondly, lower the patient load 
on the first-line-of-contact health facilities within the public sector. 
This rationale is supported by the World Health 
Organisation~\ref{WHO_AlmaMater_1970}. The South African government has 
relied heavily on not-for-profit organisations (NPO) to deploy and manage 
CCGs across the country.

The current National Community Care Giver Policy Framework 
states:
\begin{itemize}
\item The preferred operational model is a Government/NPO partnership 
where Government provides grants to NPOs, which employ and manage CCGs.
\item Volunteerism is encouraged although CCG volunteers cannot work 
more than 40 hours per month.
\item CCGs should have a support system, e.g.~be part of an NPO or community 
based organisation (CBO) and have access to health referral systems.
\item Community involvement, commitment of top management, redress of 
previous inequalities, learner contribution, stakeholder participation and 
needs-based approaches are key principles.
\item Fully trained generalist CCGs would receive a minimum stipend of R1500.
\item In rural areas each CCG would cover from 80 to 100 households, the 
corresponding number being 100 to 150 households in urban areas.
\item The district health manager has formal responsibility for monitoring 
the quality of services and providing support.
\end{itemize} 

In the revised National Community Care Giver Policy Framework, currently being 
drafted, it is estimated that by 2010 CCGs in South Africa will constitute
60\% of all health personnel with R1.8 billion of the total health budget 
being spend on the CCG programme. However, to date the South 
African government has been unable to secure the quality 
of care within the CCG programme due to the inability to standardise and monitor 
the health care service delivery to the patient. 

The current monitoring 
and evaluation (M\&E) system is paper-based and requires the CCGs to 
fill out tally sheets at the end of every month and submit these sheets 
to their NPO/CBO. The NPO (or CBO) managers or a designated M\&E officer 
has to collate the information on these tally sheets and submit an 
aggregated report of all activities to the district (or sub-district) 
programme coordinator. This same process of aggregation and reporting 
takes place on the district, provincial and national level. 

Mobile telecommunication 
technology provides an ideal tool to monitor the services provided by CCGs 
and the associate patient outcomes, and thereby allow the health service 
delivery to the patient to be standardised. In addition, the web-based 
data management interface automates the processes of aggregation 
and reporting. The primary goal of this project is to effectively 
integrate these integration and communication technologies into 
community based health services to improve basic service delivery 
on the ground.

\subsection*{System Architecture}

The CCG project architecture illustrated in Figure~\ref{fig:cbs_nompilo}. 
As previously described, CCGs are seen as the first-line-of-contact with 
the public. The system is designed such that an application is loaded onto 
a phone that enables the interaction between the CCG and the patients 
to be captured, using structured question and answers (Q\&A) trees. 
The interaction is authenticated by scanning a patient barcode using the 
camera of the mobile phone. The patient ID, GPS location and answers to 
Q\&A structures is uploaded onto a secure server that can be access by both 
NPO managers and health care facility staff so that the service provided 
by the CCGs and the patient outcomes can be monitored.  

\begin{figure} \centering 
\includegraphics[width=\textwidth]{../Figure/Nompilo/CCG_architecture.jpg}
\caption{Schematic showing the integration of the CCG project's integration 
into the public health system in South Africa [Source GeoMed (Pty) Ltd].}
\label{fig:cbs_nompilo}
\end{figure}

The stake holders and associated activities involved in the CCG project 
is illustrated in Figure~\ref{fig:nompilo_flow}. The user activities for 
each of the stakeholders is listed in Table~\ref{tbl:activities}.

\begin{tabular}{ll}
{\bf \underline{Stakeholders}} & {\bf \underline{Activities}}\\
Community Care Giver & Scan patient barcode\\
~ & Register patient details\\
~ & Enter data using Q\&A trees\\
NPO Manager & Register patients and CCGs\\
~ & Assign CCGs to patients (link barcode to patient record)\\
~ & Monitor, analyse and report aggregated data\\
Primary Health Care Facility & Register patients (link barcode to patient record)\\
~ & Monitor, analyse and report aggregated data\\
DoH Programme Coordinator & Add NPOs\\
~ & Monitor, analyse and report aggregated data\\  
\label{tbl:user_activities}
\end{tabular}

\begin{figure} \centering 
\includegraphics[width=\textwidth]{../Figure/Nompilo/CCG_processes.jpg}
\caption{Activities and data flow between various stakeholders in the  
CCG project [Source GeoMed (Pty) Ltd].}
\label{fig:nompilo_flow}
\end{figure}

The CCG project is part of a larger e-Health platform. Figure~\ref{fig:ehp} 
illustrates the integration between the CCG project and the e-Health 
Platform, as envisaged by GeoMed. The system is developed in a modular 
way such that the different services can be generalised and used in 
other applications. 

\begin{figure} \centering 
\includegraphics[width=0.6\textwidth]{../Figure/Nompilo/eHealth_platform.jpg}
\caption{Integration of the CCG project's (Nompilo) with M\&E module and 
the larger e-Health Platform, as envisaged by GeoMed and Vodacom Health 
[Source GeoMed (Pty) Ltd].}
\label{fig:ehp}
\end{figure}

\section*{Research Study}

The project proposal is to validate the feasibility, the cost effectiveness 
and scalability of employing a cellphone-based monitoring and evaluation 
solution that will a standard and integrated framework in which community 
based services provided by CCGs to the South African public can be recorded, 
effectively analysed and reported to the relevant stakeholders.

The solution will be deployed in three districts in three different provinces 
in South Africa, namely Western Cape, KwaZulu Natal and Limpopo. Three 
independent NPOs (Zanempilo Trust, The Valley Trust, CHoiCe Trust) 
have agreed to validate the scalability of the solution 
across managerial and geographical boundaries. 

The research study associated with the project will focus on the evaluating 
the impact of the electronic M\&E system. The study will combine both 
quantitative and qualitative methodologies. Firstly, cost-benefit 
analysis will be used to to assess the impact of the project costs and 
associated projects benefits. Secondly, the impact of the electronics M\&E 
system on the perception of the various stakeholders (including nurses, 
doctors, NPO managers and Department of Health CBS coordinators).  

\subsection*{Research Methodologies}

The study will utilize a participatory action research approach and 
will be conducted with the partner NPOs. Each NPO will select 20 
CCGs to participate in the study, bringing the total of number of 
participating CCGs to 60. A combination of methodologies will be 
used to monitor and evaluate the system. In the monitoring and 
evaluation phase both qualitative and quantitative information 
will be gathered. Data collection tools will be based on mobile 
phone technology but will be designed in consultation with 
the NPOs and CCGs. The study methodologies will include:
\begin{itemize}
\item A review of literature, documents, and standards of current M\&E systems
\item Focus groups with selected CCGs and community representatives  
\item Structured interviews with clinic nurses and physicians using 
structure survey on mobile phone
\item Key informant interview with stakeholders, including NPO managers, 
district programme coordinators as well as national and provincial 
programme coordinators 
\item Cost-benefit analysis of the M\&E system will 
focus on the impact of the intervention on the benefits of the M\&E 
system of CCG programme as experienced by the various stakeholders
\end{itemize}

\subsection*{Logistics}

The strategic direction for the project will be provided 
by the Research Director, Dr Irwin Friedman, at Health Systems 
Trust\footnote{An non-governmental organisation that carries out scientific 
research into health systems in South Africa.}.  The electronic/mobile 
information and communication infrastructure will be developed in 
collaboration between GeoMed (Pty) Ltd, Geo-ICT Health (Pty) Ltd and 
Pondering Panda. The implementation of the community based 
action research study will be undertaken by HST in conjunction with 
the three NPOs 

\section*{Literature Review}

Friedman depicts succinctly the historical background of the 
use of community based health workers (referred to as community 
care givers in this document) from the time when World Health 
Organization (WHO) declared Alma Ata in 1978~\cite{WHO1978_Alma_Ata} 
and established the Primary Health Care (PHC) 
paradigm~\cite{Friedman2002_CHWs}. He further mentions that in one form 
or other the CBHWs have worked in different parts of South Africa, taking 
on varied responsibilities and roles in Home and Community Based 
Care (HCBC) activities. Non-governmental organizations (NGOs) in 
particular were early to recognize their worth in extending PHC services.  

Relying on the strengths of family and community networks, 
Home and Community Based Care (HCBC) has emerged as an effective 
method of providing cost-effective public health initiative, 
particularly for people living with HIV/AIDS 
(PLWHA). HCBC is not a replacement for hospital 
care, but instead is part of a comprehensive continuum of prevention, 
care, treatment, and support services that include the family, the 
community, and various levels of health care providers. In addition to 
providing palliative and curative care, HCBC also 
contributes to prevention efforts by involving community members in 
prevention, care, and support efforts.

In I994, the South African Cabinet mandated the Departments of Social
Development and Health to oversee the implementation of HCBC and
support programme in the country. Subsequently in 2005, studies were
conducted on a small scale and with a limited sampling size to
evaluate a few components of the HCBC such as the audit of caregivers
and the evaluation of costs and process indicators for HCBC including
two appraisals of HCBC that were conducted in 2003/2004 and 2004/2005
financial years respectively. More recently, a study has been
completed in the evaluation of M\&E systems within HCBC with the view
to establish an integrated M\&E system.

In June, 2007 the National Department of Health (NDOH) through the
European Union (EU) funded project – Partnerships for the Delivery of
Primary Health Care including HIV and AIDS Programme (PDPHCP)
undertook a literature review on NPO M\&E Systems in place in South
Africa. The review revealed that there is no standardized M\&E
Framework for NPOs and to make it operational across the board will
take time. Subsequent to the review came two documents (a) Monitoring
and Evaluation Systems for NPOs “Guidelines, Dataflow Policy, Care
Packages and Routine Data Collections Tools and (b) Monitoring and
Evaluation Framework for Non-Profit Organizations (NPOs) Delivering
Primary Health Care Services. 

There is a lot of variation in methodology in evaluating the 
impact of mobile health (m-health) projects~\cite{Vital2008_mHealth_4Dev}. 
Furthermore, there are few studies that have used economic 
evaluation to assess impact of the proposed m-Health 
intervention. In a recent review by Whitten et al.~concluded that 
``there is no good evidence  to suggest that telemedicine is 
a cost effective of delivering health 
care''~\cite{Whitten2002_telemedicine_CEA}. A recent evaluation of 
ten European e-Health applications suggested that 
cost-benefit analysis is the best way to asses the multiple 
and varied outcomes of e-Health 
solutions~\cite{Stroetmann2007_EU_eHealth_rev}.
A cost-benefit analysis allows the varied benefits experienced by 
the various stakeholders to be quantified and incorporated into the 
analysis~\cite{Drummond1977_health_economics}. 

\section*{Evaluation Model}

The objective of the evaluation model is to be a tool to assess 
the impact of the CCG project by combining the quantitative and 
qualitative resulting from the study. The quantitative portion 
of the evaluation model is a cost-benefit model that 
records the costs and quantifies the benefits associated with the 
implementation of the solution. It is critical that 
the potential benefits in improved access, quality of care, and 
better clinical results are clearly demonstrated. 

\subsection*{Cost-benefit Analysis}

A cost-benefit allows the enables the impact on all stakeholders to be 
included in the evaluation, so long as the range of benefits can be 
expressed in monetary terms. The objective of the cost-benefit analysis 
is to determine the total value added and the associated net benefit 
of the proposed intervention. In this instance, the costs and benefits 
will be assess for the current paper-based solution and this will 
be compared with the costs and benefits associated with intervention.

The first step in cost-benefit analysis is to identify the various 
stakeholders who will be beneficiaries from the proposed intervention. 
Furthermore, these stakeholders can be classified as primary or 
secondary stakeholders. Secondary stakeholders are those stakeholders 
that only experience indirect benefits as opposed to direct benefits.
For the CCG project the primary stakeholders are, in no particular order, 
the CCGs, NPO managers, programme coordinators at sub-district, 
district, provincial and national level of the departments of 
health and social development. Secondary stakeholders include 
patients, public health service providers (eg.~nurses and doctors), 
the department of labour and national treasury.

The direct benefits associated with the CCG project are saved time, saved 
costs and improved data accuracy. Data accuracy can be broken down into 
the extent of data captured as well as the time it takes to capture 
and access the data at various level of centralisation. 
Table~\ref{tbl-benefits} lists the benefits and the beneficiaries of the 
proposed intervention.

\begin{table} \centering
\begin{tabular}{l|c|c|c|c|c|c|}
{\bf Benefit} & CCG & NPOs & PHC & DoH & DoSD & Treasury \\ \hline
Saved time & $\checkmark$ & $\checkmark$ & ~ & $\checkmark$ & $\checkmark$ & ~ \\ 
Saved costs & $\checkmark$ & $\checkmark$ & ~ & $\checkmark$ & $\checkmark$ & $\checkmark$ \\ 
Data accuracy & ~ & $\checkmark$ & $\checkmark$ & $\checkmark$ & $\checkmark$ &  ~  \\ 
\end{tabular}
\caption{Benefits and the associated beneficiaries.}
\label{tbl-benefits}
\end{table}
   
The benefits need to be quantified and monetised so that they can be 
included in the cost-benefit analysis. Table~\ref{tbl:benefits_monetisation} 
lists the different benefits and how these benefits could be monetised.

\begin{table} \centering
\begin{tabular}{|l|l|l|} \hline
{\bf Benefit} & {\bf Quantification} & {\bf Monetisation} \\ \hline
Saved time & $\cdot$ Time spend doing M\&E & $\cdot$ \% work hours times wage\\
~ & $\cdot$ Time spent travelling & ~\\
~ & $\cdot$ Time spend reporting & ~\\ 
~ & ~~or processing feedback & ~\\ \hline
Saved costs & $\cdot$ Transport costs & $\cdot$ Taxi/bus fares\\
~ & $\cdot$ M\&E operational costs & $\cdot$ Photocopying, faxing costs etc.\\
~ & $\cdot$ Overhead costs & $\cdot$ Administrative or data capturing costs\\ \hline
Data accuracy & $\cdot$ Number of ``correctly''  & $\cdot$ Number of client visit times \\
~ & ~~recorded client visits & ~~CCG remuneration per client\\
~ & $\cdot$ Lag time to data submission & $\cdot$ Willingness to pay (WTP) for\\~ & ~ & ~~early warning divided by lag time\\\hline 
\end{tabular}
\caption{Monetisation of M\&E costs and associated benefits.}
\label{tbl:benefits_monetisation}
\end{table}

The costs of the CCG project can be grouped into investment costs and 
operational costs. The investment costs would include system design and 
development, as well as any capital expenditure in order to implement the 
solution. Training and change management costs would also be classified 
as an investment cost. Operation costs on the other hand would include 
the costs associated with system maintenance, user support, data transfer 
costs, server hosting and back-end services.
 
The present values are calculated for the annual and cumulative 
costs and benefits. The present values are determined using a simple 
discounted cash flow technique. The net benefit is determine for both the 
present and cumulative values as the difference between the total  
benefits and the total costs. Furthermore, the distribution of the 
benefits amongst the stakeholders can also be determined.
The outcome of the cost-benefit analysis is to be able to assess the 
costs and benefits of the intervention over time and be able to identify 
when the cumulative benefits exceed the costs.
   
\section*{Conclusion}

The project is scheduled to be implemented across the three pilot 
sites starting on 1 November 2009 and running for six months until 
the end of April 2010. Obviously, the outcome of the cost-benefit 
analysis is yet to be determined.  